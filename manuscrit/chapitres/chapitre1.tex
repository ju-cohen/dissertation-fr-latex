\chapter{Titre du chapitre 1}
\label{chap1}
\begin{justify}
Résumé du chapitre 1.
\end{justify}
\adjustmtc  % ajuste la numérotation
\minitoc
\newpage

\section{Titre}
\lipsum[1-3]
%
\\
\indent Texte avec un renvoi vers le pied de page~\footnote{\url{https://eidolon.univ-lyon2.fr/wiki/index.php/Accueil}}.

\begin{figure}[h]
\centering
	\begin{subfigure}[t]{0.4\textwidth}
		\centering
		\includegraphics[width=\textwidth]{chapitre1/Logo}
		\subcaption[caption]{Logo 1~\footnotemark{}.}  % [caption] nécessaire pour que le \footnotemark{} fonctionne
	\end{subfigure}
	\hfill
	\begin{subfigure}[t]{0.4\textwidth}
		\centering
		\includegraphics[width=\textwidth]{chapitre1/Logo}
		\subcaption[caption]{Logo 2~\footnotemark{}.}
	\end{subfigure}
\caption{Plusieurs footnotes avec url dans des légendes et la galère de les numéroter correctement.}
\end{figure}
% url dans une légende : utiliser footenotemark dans la légende + footnotetext après la figure
% Si plusieurs footnotes : le compteur s'est incrémenté plusieurs fois, il faut l'ajuster manuellement
\addtocounter{footnote}{-1}  % recule le compteur (qui a avancé avec les marques)
\footnotetext{footnote associée à la première marque de la figure}  % contenu du footnote
\stepcounter{footnote}  % avance le compteur de 1
\footnotetext{marque de la 2e figure: \url{https://eidolon.univ-lyon2.fr/wiki/index.php/Accueil}}  % contenu du footnote
