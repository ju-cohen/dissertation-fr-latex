\documentclass[12pt,a4paper,twoside,openright]{report} % default margins: 1.5in
% twoside + openright : nouveaux chapitres toujours sur la page de droite
\usepackage[utf8]{inputenc}
\usepackage[T1]{fontenc}
\usepackage{natbib}
\usepackage[frenchb]{babel}
\usepackage{ragged2e}
\usepackage[margin=2.5cm]{geometry}
\usepackage{amsmath}
\usepackage{amsfonts}
\usepackage{amssymb}
\usepackage{graphicx}
\graphicspath{{./figures/}}  % pour indiquer où trouver les figures
\usepackage{pifont}
\usepackage{caption}
\usepackage{subcaption}
\usepackage{titlesec}  % formattage des titres
\usepackage{enumitem}  % personnalisation des listes
\usepackage[bottom]{footmisc}  % placer les footnotes en bas de page et pas juste après le paragraphe
\usepackage{pgfplots}
\usepackage{pgfplotstable}
\pgfplotsset{compat=1.7}
\pgfplotstableset{col sep=comma}  % pour lire des csv 
\usetikzlibrary{patterns}
\usepackage{arydshln}  % pour hachurer des zones dans les éléments tabular et arrays
\usepackage{multirow}  % pour ajuster le contenu des tableaux
\usepackage[Algorithme]{algorithm} % entre crochets = le nom qui s'affiche pour identifier l'algo.
\usepackage{algpseudocode}
\usepackage[french]{minitoc}  % pour des tables des matières partielles en début de chapitre
\usepackage{placeins}
\usepackage[header]{appendix}
\usepackage{calrsfs}
\usepackage{fancyhdr}  % pour la gestion des en-têtes et pieds de pages
\usepackage{eurosym}
\usepackage{siunitx}  % pour aligner les nombres dans les tableaux (à la virgule, etc.), avec S au lieu de c
\sisetup{output-decimal-marker = {,}}
\usepackage{remreset}  % pour avoir des numéros de notes de bas de pages continus, pas réinitialisés à chaque chapitre (avec la ligne suivante)
\makeatletter\@removefromreset{footnote}{chapter}\makeatother

\usepackage{lipsum}  % pour le template uniquement

\PassOptionsToPackage{hyphens}{url}\usepackage{hyperref}  % doit être le dernier chargé

\pagenumbering{roman}  % pour que les premières pages soient numérotées en chiffres romains

\setcounter{secnumdepth}{4}  % numéroter jusqu'au niveau de profondeur 4
\setcounter{tocdepth}{1}
\renewcommand{\theparagraph}{\alph{paragraph})} % numérotation a), b)... pour niveau \paragraph
\renewcommand{\frenchtablename}{Tableau}  % pour remplacer l'identifiant par défaut "Table"

\newcommand{\xmark}{\ding{55}}  % Opposé du symbole \check

\renewcommand{\algorithmicrequire}{\textbf{Entrées :}}  % Renommage de mots-clés
\renewcommand{\algorithmicensure}{\textbf{Sorties :}}  % Renommage de mots-clés

\renewcommand{\appendixname}{Annexe}  % Renommage de mots-clés
\renewcommand{\appendixtocname}{Annexes}  % Renommage de mots-clés
\renewcommand{\appendixpagename}{Annexes}  % Renommage de mots-clés

\pagestyle{fancy}  % 
\fancyhf{}
\fancyhead[LE]{\small \leftmark}  % champ gauche + page paire
\fancyhead[RO]{\small \rightmark}  % champ droit + page impaire
\fancyfoot[C]{\thepage}  % pied de page au centre

\renewcommand{\headrulewidth}{1pt}

\begin{document}
% Huge \huge \LARGE \Large \large \normalsize \small \footnotesize \tiny
\begin{titlepage}
\begin{center}
%
\includegraphics[width=0.3\linewidth]{Logo}
%
\\
\vspace*{1cm}
%
{\Large
Université Lumière Lyon 2
} \\
%\vspace*{1cm}
%
{\large
\textbf{Ecole Doctorale : ED 512 Informatique et Mathématiques} \\
\textit{LIRIS}\\
}
%
\vspace*{1cm}
{\Large
\MakeUppercase{Thèse}\\
pour obtenir le grade de\\
\MakeUppercase{Docteur en informatique}\\
}
\vspace*{1.5cm}
{\LARGE
\textbf{TITRE}
} \\
%
%\vspace{0.5cm}
%Thesis Subtitle
%
\vspace{1.5cm}
%
{\Large
AUTEUR \\
}
%
\vspace{1.5cm}
%
{\large
Informatique \\
}
%
\vspace{1cm}
{\normalsize
Sous la direction de DIRECTEUR(S) DE THESE \\
Co-encadrée par CO-ENCADRANT(S) \\
}
%
\vfill
%
{
Soutenue publiquement le DATE \\
}
%
\end{center}
%
\vspace{0.5cm}
\noindent Composition du jury : \\
Prénom NOM, Titre, Université \\

\end{titlepage}





\clearpage
\thispagestyle{plain}  % remove header rule on blank pages, but not page number

\dominitoc  % Initialisation des minitoc

\chapter*{Remerciements}


\newpage
\thispagestyle{plain}
\chapter*{Résumé}  
Résumé en français.
%
\newpage
\thispagestyle{plain}

\chapter*{Abstract}
Résumé en anglais.
%
\newpage
\thispagestyle{plain}
\tableofcontents
\newpage

\pagenumbering{arabic}  % Débute la numérotation normale des pages
\setcounter{page}{1}
\setcounter{section}{1}

\chapter*{Introduction générale}
\addcontentsline{toc}{chapter}{Introduction générale}  % Nécessaire car chapitre non numéroté
\label{intro}
\markboth{\MakeUppercase{Introduction générale}}{}  % Ajuste l'en-tête
\lipsum

\clearpage
\chapter{Titre du chapitre 1}
\label{chap1}
\begin{justify}
Résumé du chapitre 1.
\end{justify}
\adjustmtc  % ajuste la numérotation
\minitoc
\newpage

\section{Titre}
\lipsum[1-3]
%
\\
\indent Texte avec un renvoi vers le pied de page~\footnote{\url{https://eidolon.univ-lyon2.fr/wiki/index.php/Accueil}}.

\begin{figure}[h]
\centering
	\begin{subfigure}[t]{0.4\textwidth}
		\centering
		\includegraphics[width=\textwidth]{chapitre1/Logo}
		\subcaption[caption]{Logo 1~\footnotemark{}.}  % [caption] nécessaire pour que le \footnotemark{} fonctionne
	\end{subfigure}
	\hfill
	\begin{subfigure}[t]{0.4\textwidth}
		\centering
		\includegraphics[width=\textwidth]{chapitre1/Logo}
		\subcaption[caption]{Logo 2~\footnotemark{}.}
	\end{subfigure}
\caption{Plusieurs footnotes avec url dans des légendes et la galère de les numéroter correctement.}
\end{figure}
% url dans une légende : utiliser footenotemark dans la légende + footnotetext après la figure
% Si plusieurs footnotes : le compteur s'est incrémenté plusieurs fois, il faut l'ajuster manuellement
\addtocounter{footnote}{-1}  % recule le compteur (qui a avancé avec les marques)
\footnotetext{footnote associée à la première marque de la figure}  % contenu du footnote
\stepcounter{footnote}  % avance le compteur de 1
\footnotetext{marque de la 2e figure: \url{https://eidolon.univ-lyon2.fr/wiki/index.php/Accueil}}  % contenu du footnote


\clearpage
%Conclusion
\chapter*{Conclusion générale et perspectives}
\label{conclusion}
\addcontentsline{toc}{chapter}{Conclusion générale}
\markboth{\MakeUppercase{Conclusion générale}}{}

\lipsum

\chapter*{Annexes}
\addcontentsline{toc}{chapter}{Annexes}
\markboth{\MakeUppercase{Annexes}}{}
\adjustmtc[2]  % Ajuster les numéros de pages
\minitoc
\newpage

\begin{subappendices}
% Numérotation des sections, sous-sections et figures
\renewcommand{\setthesection}{\Alph{section}}
\renewcommand{\setthesubsection}{\Alph{subsection}}
\renewcommand\thefigure{\thesection.\arabic{figure}}
\setcounter{figure}{0}  

\addtocontents{toc}{\protect\setcounter{tocdepth}{-1}}
\section{Titre de mon annexe A}
\label{annexe:A}

\subsection*{Sous-titre non numéroté}

\newpage
\section{Titre de mon annexe B}
\label{annexe:B}

\subsection*{Sous-titre non numéroté}

%
\begin{figure}[!h]
\centering
	\includegraphics[width=0.3\textwidth]{Logo}
\caption[Titre court]{Figure pour montrer la numérotation en annexe.}
\end{figure}
%

\end{subappendices}
\addtocontents{toc}{\protect\setcounter{tocdepth}{1}}

\bibliographystyle{plainnat-fr}
\bibliography{references}
\addcontentsline{toc}{chapter}{Références}  % ajouter les références dans la table des matières

\listoffigures
\addcontentsline{toc}{chapter}{Table des figures}
\listoftables
\addcontentsline{toc}{chapter}{Liste des tableaux}

\end{document}